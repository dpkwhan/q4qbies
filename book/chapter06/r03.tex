\section{Zero Padding}

The RIC (Reuters Instrument Code) code for stocks traded in Hong Kong Stock Exchange has \textbf{four} digits before the dot. For example, the RIC code for HSBC is \q{0005.HK}. Given the snippet of a table like below, we want to add another column called \q{ric} to this table.

\begin{minted}[samepage,frame=single,framesep=10pt,xleftmargin=10pt]{q}
q) master:([]sym:`5`11;code:`HK);
q) master
sym code
--------
5   HK
11  HK
\end{minted}

After adding the \q{ric} column, the table looks like:
\begin{minted}[samepage,frame=single,framesep=10pt,xleftmargin=10pt]{q}
sym code ric
----------------
5   HK   0005.HK
11  HK   0011.HK
\end{minted}

\subsection{Implementation}
Three different approaches are suggested to zero pad the \q{sym} with less than $4$ digits.

\begin{minted}[samepage,frame=single,framesep=10pt,xleftmargin=10pt,linenos]{q}
master:update ric:{` sv x} each flip (sym;code) from master;

// Implementation #1
update {`$"0"^-7$x} each string ric from master

// Implementation #2
update {`$-7#"000",x} each string ric from master

// Implementation #3
update {`$count[x]_"0000000",x} each string ric from master
\end{minted}

\subsection{Explanations}
\begin{itemize}
\item Line 1 adds a new column \q{ric} by concatenating the \q{sym} and \q{code} with a dot. Note that \q{x} in the anonymous function is a two-element symbol list and with null symbol (\q{`}) as the left argument, \q{sv} concatenates elements of symbol list using a dot. This is still not what we want yet since the values for \q{sym} column with less than 4 digits is not zero padded to have a total of 4 digits. 
\item Line 4 first pre-appends spaces so that the length of the resulting string is exactly 7 and then uses \q{"0"} replace the spaces with operator (\q{^}). Note that the coalesce operator (\q{^}) replaces the null values in the right argument with values from the left argument and the null value of \q{string} is a single space (\q{" "}). 
\item Line 7 first pre-appends three \q{0}'s so that the length of the resulting string is not less than 7 and then the last 7 characters are kept.
\item Line 10 first pre-appends seven \q{0}'s and drops the extra \q{0}'s in the front so that the length of the resulting string is 7.
\end{itemize}


\subsection{Summary}

\begin{noteblock}
\textbf{Knowledge Points}
\begin{itemize}
\item Operators: \href{https://code.kx.com/q/ref/coalesce/}{\q{^}}, \href{https://code.kx.com/q/ref/drop/}{\q{_}} and \href{https://code.kx.com/q/ref/take/}{\#}
\item Operator \q{$}: \href{https://code.kx.com/q/ref/cast/}{Type casting} and \href{https://code.kx.com/q/ref/pad/}{Pad}
\item Functions: \href{https://code.kx.com/q/ref/count/}{\q{count}}, \href{https://code.kx.com/q/ref/string/}{\q{string}}, \href{https://code.kx.com/q/ref/each/}{\q{each}} and \href{https://code.kx.com/q/ref/flip/}{\q{flip}}
\end{itemize}
\end{noteblock}

\subsection{Extras}
\begin{tipblock}
A convenient function \href{https://code.kx.com/q/ref/dotq/#qdd-join-symbols}{\q{.Q.dd}} makes joining symbols easier. For example, you can use the following to replace Line 1:
\begin{minted}[frame=lines,framesep=10pt,xleftmargin=10pt]{q}
update ric:.Q.dd'[sym;code] from master
\end{minted}

Another example:
\begin{minted}[frame=lines,framesep=10pt,xleftmargin=10pt]{q}
q) .Q.dd[`IBM]"N"
`IBM.N
q) .Q.dd[`IBM]`N
`IBM.N
\end{minted}
\end{tipblock}

\clearpage
