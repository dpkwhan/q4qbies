\section{Commafy Numbers}

In English, commas are used to separate numbers with more than three digits in the whole-number part. A comma is used every third digit from the right before the decimal point if there is one.

\subsection{Implementation}

\begin{minted}[samepage,frame=single,framesep=10pt,xleftmargin=10pt,linenos]{q}
.format.commafy:{
  s:"." vs string abs x;
  r:reverse","sv 0N 3#reverse s 0;
  r:$[2=count s;"." sv (r;s 1);r];
  $[x<0;"-",r;r]
  };
\end{minted}

\subsection{Examples}
Here are a few examples of how to use this function:

\begin{minted}[frame=lines,framesep=10pt,xleftmargin=10pt]{q}
q) .format.commafy -1234.56
"-1,234.56"

q) .format.commafy 1234567
"1,234,567"

q) .format.commafy 1.2e5
"120,000"
\end{minted}

\subsection{Explanations}
\begin{itemize}
\item Line 2 first convert the absolute value of the input number to string format and split the number by dot, \emph{i.e.} separating the whole-number part and decimal part.
\item Line 3 reverses the order of all digits in the whole-number part and splits them into trios, \emph{i.e.} each group has three digits except the last group if the total number of digits is not a multiple of 3. It then concatenates all trios with a comma. Finally it reverses the order back to their original order.
\item Line 4 checks whether there is a decimal point in the orginal number. If so, combine the whole-number part and decimal part. 
\item Line 5 adds back the negative sign if the original input number is negative.
\end{itemize}

\subsection{Summary}
\begin{noteblock}
\textbf{Knowledge Points}
\begin{itemize}
\item Operators: \href{https://code.kx.com/q/ref/cond/}{\q{$}} and \href{https://code.kx.com/q/ref/cut/}{\#}
\item Functions: \href{https://code.kx.com/q/ref/abs/}{\q{abs}}, \href{https://code.kx.com/q/ref/string/}{\q{string}}, \href{https://code.kx.com/q/ref/vs/}{\q{vs}}, \href{https://code.kx.com/q/ref/sv/}{\q{sv}}, \href{https://code.kx.com/q/ref/count/}{\q{count}} and \href{https://code.kx.com/q/ref/reverse/}{\q{reverse}}
\end{itemize}
\end{noteblock}

\clearpage
