\section{Identity Matrix}
\label{example:LinearInterpolcation}

In linear algebra, an identity matrix of size 3 looks like:

$$
\left[
\begin{matrix}
1 & 0 & 0\\
0 & 1 & 0\\
0 & 0 & 1\\
\end{matrix}
\right]
$$


\subsection{Implementation}
Below is the function to create an identity matrix of any given size $n$.

\begin{minted}[samepage,frame=single,framesep=10pt,xleftmargin=10pt,linenos]{q}
.qe.math.identity:{[n]
  `float$v=/:v:til n
  };
\end{minted}

And the output \q{.qe.math.identity[4]} is

\begin{verbatim}
1 0 0 0 
0 1 0 0 
0 0 1 0 
0 0 0 1 
\end{verbatim}


\subsection{Explanations}
Let's look at this function in details.

\begin{itemize}
\item \begin{minted}{q}
v:til n
\end{minted}
\q{til n} returns all integers from zero up to $n$, but not including $n$ and assigns the list of integers to variable \q{v}. Note that the assignment operator returns the result of the expression assigned to $v$, which is later used by other expression to the left.

\item \begin{minted}{q}
v=/:v
\end{minted}
The list \q{v} is compared with each element of itself. So the returned value is a list of list of boolean values. For example, when \q{v} is compared with $v_i$ where $i = 0, 1, \cdots, n-1$, a list of boolean values is returned and the $i^{th}$ element is \q{1b} and \q{0b} elsewhere. Let $m_{ij}$ denote the element $j$ of the inner list, which is the element $i$ of the outer list, then 

$$
m_{ij} = 
\begin{cases}
\enspace \q{1b} & \text{ if } i=j, \\
\enspace \q{0b} & \text{ else. }
\end{cases}
$$

\item The result is an $n$x$n$ matrix of boolean values, which is then casted to a matrix of floating point numbers. The boolean value \q{1b} is casted to \q{1f} and \q{0b} is casted to \q{0f}.
\end{itemize}


\subsection{Summary}

\begin{importantblock}
\textbf{Important Note}
\begin{itemize}
\item For some matrix operators like \q{mmu}, \q{inv} and \q{$}, the matrix entries must of \q{float} type, \emph{i.e.} \q{all 9h=type each m} must be true if \q{m} is a matrix.
\end{itemize}
\end{importantblock}

\begin{noteblock}
\textbf{Knowledge Points}
\begin{itemize}
\item Function \href{https://code.kx.com/q/ref/til/}{\q{til}} 
\item Return value of assignment operation\footnote{See section Syntax on this page: \url{https://code.kx.com/q/ref/assign/}}
\item Each right \href{https://code.kx.com/q/ref/maps/#each-left-and-each-right}{\q{/:}}
\item Type casting: \href{https://code.kx.com/q/ref/cast/}{\q{$}}
\item Matrix entries must be of \q{float} type
\end{itemize}
\end{noteblock}

\clearpage
